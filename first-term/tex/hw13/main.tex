\documentclass{article}

\usepackage[utf8]{inputenc} 
\usepackage[russian]{babel} 
\usepackage{amsmath} 
\usepackage{hyperref} 
\usepackage{multicol}
\usepackage{graphicx}
\usepackage{amssymb}
\usepackage{tipa}

\title{Домашнее задание по АиСД №13}
\date{2019-12-20}
\author{Эмиль Гарипов M3138}

\begin{document}
\pagenumbering{gobble}

\maketitle
\newpage
\pagenumbering{arabic}

\section*{Задача №4}
Докажем по индукции. 
Индукционное предположение: 
для всех листовых вершин, находящихся на высоте не более $n$ верно следующее: $\sum_{i = 1}^{cnt}{2^{-d_i}} \leq 1$, где $cnt$ "--- кол-во листовых вершин на высоте не более $n$, а $d_i$ "--- высота $i$-ой вершины. 

\begin{enumerate}
	\item $\textbf{База n = 1}$.
	Для одной вершины, находящейся на высоте 1 утверждение верно, $ 1 \leq 1$.
	Далее будем <<строить>> имеющееся дерево. База верна для листьев, тогда возьмем очередной лист и подвесим его к его родителю в исходном дереве. Так же, если у листа имеется брат, подвесим его аналогчным образом. Тогда лист будет иметь высоту 2, родитель высоту 1, а утверждение для $n = 2$ будет верно в силу доказанного ниже перехода. Затем продолжим такое построение до тех пор, пока не дойдем до корня. Дойдя до корня докажем утверждение для всего дерева.
	\item $\textbf{Переход. Утв. верно для n, докажем для n + 1}$. Возьмем вершину, назовем ее $a$, находящуюся на высоте $n$. Подвесим ее вместе с ее братом(если он имеется), назовем его $b$, к их отцу в исходном дереве. Для поддерева вершины $a$ и $b$(если вершины $b$ не существует, тогда будем считать, сумму из индукционного предположения равной нулю) верно по предположение индукции следующее:
	\begin{equation}
	\sum_{i = 1}^{cnt_a}{2^{-da_i}} \leq 1 , \text{и } \sum_{i = 1}^{cnt_b}{2^{-db_i}} \leq 1
	\end{equation}\ 
	где $cnt_a$ "--- количество листовых вершин в поддереве вершины $a$, $cnt_b$ "--- количество листовых вершин в поддереве вершины $b$, $da_i$ и $db_i$ "--- высоты листовых вершины в поддеревьях вершин $a$ и $b$ соответственно.\\\\
	Сложим эти две суммы и поделим на два. Получим, что утверждение верно и для (n + 1). Так как при подвешивании вершин $a$ и $b$ к их отцу мы увеличиваем высоту листовых вершин их поддеревьев на один. А деля на два сумму двух сумм, мы выносим за скобку $0.5$, что соотвествует увеличению высоты каждого листа из поддеревьев вершин $a$ и $b$ на один. Индукционный переход доказан.
\end{enumerate}
В силу произвольности дерева утверждение, которое необходимо доказать в задаче, верно для любого двоичного дерева.
\section*{Задача №5} 
Зафиксируем начальную вершину, назовем ее $a$, и конечную вершину, на которой мы остановимся, назовем ее $b$. Нетрудно заметить, что в ходе выполнения операции $veryNext$ мы посетим все вершины на путях от $a$ до $lca(a, b)$ и от $lca(a, b)$ до $b$. Таких вершин не более $2 \cdot \log_{2}{n}$. А все остальные вершины(которые не лежат на путях от $a$ до $lca(a, b)$ и от $lca(a, b)$ до $b$), которые мы посетим, их не более $\mathcal{O}(k)$, причем мы сначала в какой-то момент зайдем во все такие вершины, потом выйдем из них, значит посетим каждую два раза, следовательно посещение таких вершин займет $\mathcal{O}(k)$ времени. Отсюда получаем время работы $\mathcal{O}(k + \log_{2}{n})$.
\end{document}