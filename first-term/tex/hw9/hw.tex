\documentclass{article}

\usepackage[utf8]{inputenc} 
\usepackage[russian]{babel} 
\usepackage{amsmath} 
\usepackage{hyperref} 
\usepackage{multicol}
\usepackage{graphicx}
\usepackage{amssymb}
\usepackage{tipa}

\title{Домашнее задание по АиСД №9}
\date{2019-11-22}
\author{Эмиль Гарипов M3138}

\begin{document}
\pagenumbering{gobble}

\maketitle
\newpage
\pagenumbering{arabic}

\section*{Задача №1}
Задана последовательность $a$ длины $n$, индексация с нуля.\\
Введем функцию $dp_i$ "--- кол-во различных подпоследовательностей на префиксе длины $i + 1$. Тогда:
\begin{enumerate}
	\item[] {$dp_0$ = 1} 
	\item[] {$$
		dp_i =\begin{cases}
		2 * dp_{i - 1} + 1,&\text{если элемент $a_i$ еще не встречался на префиксе длины $i + 1$;}\\
		2*dp_{i - 1} - dp_{j - 1},&\text{если элемент $a_i$ встречался на префиксе длины $i + 1$}
		\end{cases}
		$$
		, где $j$ "--- самая правая позиция(исключая $i$-ю позицию) на префиксе длины $i + 1$, в которой стоит элемент $a_i$. 
	
}
\end{enumerate}
 

\subsection*{Коррестность}
Рассмотрим пересчет значений. Для доказательства корректности пересчета, необходимо показать, что считаются все подпоследовательности на префиксе, и каждая считается ровно один раз.\\
В первом случае (когда элемент $a_i$ еще не встречался на префиксе), мы просто берем все подпоследовательности, которые есть на префиксе($dp_{i - 1}$), а так же добавляем к этим подпоследовательностям элемент $a_i$,(таких подпоследовательностей еще $dp_{i - 1}$) и добавляем подпоследовательность, состоящую из одного элемента "--- элемента $a_i$. Таким образом, полагаясь на то, что значения динамики на префиксе посчитаны верно, мы посчитали все возможные подпоследовательности, причем каждую ровно 1 раз.\\
Во втором случае (когда элемент $a_i$ встретился последний раз в позиции $j$), мы добавляем к значению $dp_i$ кол-во последовательностей на префиксе длины $i - 1$, так же добавляем к этим подпоследовательностям элемент $a_i$(таких подпоследовательностей еще $dp_{i - 1}$). Но теперь некоторые подпоследовательности мы посчитали два раза. А именно те, к которым мы могли либо дописать элемент $a_i$, либо не дописать при подсчете динамики на префиксе. Их количество "--- $dp_{j - 1}$.  Итого, получаем формулу, написанную выше. 
\subsection*{Реализация}
На очередной итерации алгоритма будем обновлять самую правую позицию, в которой встречался элемент. Просто заведем массив $pos$, длиной $n$. При встрече элемента $x$ обновим $pos_x := i$.\\
Это позволит нам реализовать алгоритм за $\mathcal{O}(n)$ сложений и вычитаний и $\mathcal{O}(n)$ других операций.


\section*{Задача №3}
Для решения задачи будем строить НВП алгоритмом за $\mathcal{O}(n\log{n})$. Таким же образом, только идя с конца, будем строить НУП(наибольшую убывающую подпоследовательность). Дополнительно при пересчете динамики, будем запоминать куда мы поставили $i$-й  элемент(т.е. на какое место мы пытались поставить $i$-й элемент в НВП(НУП)). Получим два массива $p1$ и $p2$ - описанный массив для НВП на префиксе и НУП на суффиксе.
\\
Построив массив, можно посчитать ответ. Если $p1_i + p2_i + 1 = n$ (индексация в НВП с нуля), то элемент $a_i$ может стоять в НВП на позиции $p1_i$. Теперь, для каждой длины запомним, сколько элементов могут стоять на этой длине. Если для какой-то длины всего один такой элемент, то этот элемент входит во все НВП. Если не один, то входит хотя бы в одну.
\end{document}