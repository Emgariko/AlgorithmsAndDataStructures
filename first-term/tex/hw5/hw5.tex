\documentclass{article}

\usepackage[utf8]{inputenc} 
\usepackage[russian]{babel} 
\usepackage{amsmath} 
\usepackage{hyperref} 
\usepackage{multicol}
\usepackage{graphicx}
\usepackage{amssymb}
\usepackage{tipa}

\title{Домашнее задание по АиСД №5}
\date{2019-10-25}
\author{Эмиль Гарипов M3138}

\begin{document}
\pagenumbering{gobble}

\maketitle
\newpage
\pagenumbering{arabic}

\section*{Задача №1}
Требуется доказать утверждение, назовем его $\textbf{Утверждение 1}$, что в любой сортирующей сети для любых двух соседних входных позиций $i$, $i + 1$ найдется компоратор, который их сравнивает.
Предположим, что существует сортирующая сеть, в которой нет компоратора между какой-то парой позиций $(i, i + 1)$ нет компаратора. Тогда если это предположение не верно, $\textbf{Утверждение 1}$ доказано.
\\\\Занумеруем нити(входы, на который подаются элементы) от $1$ до $n$. По нашему предположению сеть сортирующая, значит она сортирует любую последовательность из $0$ и $1$. Тогда при подаче на нити $1, 2 \ldots (i - 1)$ нулей, а на все остальные нити единичек, сортирующая сеть не сможет отсортировать эту последовтельность, так как любой компаратор между парой нитей $(i, x) \forall x \neq (i + 1)$ не изменит значение на $i$-й нити (поскольку на ней стоит 1, на всех нитях с номером меньшим $i$ стоят 0, а на нитях с номером, большим $i + 1$ стоят 1, а так же нет компоратора $(i, i + 1)$). Аналогично рассматривается случай для $i + 1$-й нити(случаи отличаются только значениями на нитях).\\ Тогда наше предположение неверно. Следовательно $\forall i < n$ в сортирующей сети $\exists comp(i, i + 1)$, где $comp(i, j)$ "--- компаратор между парой нитей $(i, j)$.

\section*{Задача №2}
Сеть должна объединять входную отсортированную последовательность $a$ и 1 элемент. Это значит, что независимо от входной последовательности и элемента, назовем его $x$, она должна ставить этот элемент на такое место, при постановке на которое верно, что выходной массив будет отсортированным (всего таких мест при разных входных данных может быть $n$). Ведь в противном случае, можно подобрать такие входные данные, которые не будут объединены сетью, а значит сеть не будет удовлетворять нужному нам условию. 

Итого, сеть должна ставить элемент $x$ на одно любое(в силу произвольности входных данных в общем случае) из $n$ мест в зависимости от входных данных. Назовем доказанное утверждение $\textbf{Утверждение 2}$.
\\\\
Докажем по индукции, что сеть глубной $i$ может ставить элемент $x$ на $2^i$ различных мест(под местами подразумеваются выходные нити сети) в завивимости от входных данных. 
\begin{enumerate}
\item $\textbf{База} \ i = 1$: Сеть глубиной 1. На входе имеется некая отсортированная последовательность $a$ и элемент $x$. Очевидно, что компаратор, связывающий два элемента последовательности не поменяет их местами, это понятно из принципа работы компаратора, ведь он отправляет на верхний выход минимум их входов, а на нижний "--- минимум (на нижний вход мы подадим элемент меньший элемента на верхнем входе в силу того, что массив отсортирован). А компараторы, связывающие элемент $x$ и элемент последовательности $a$ могут поменять их местами, но такой компаратор может быть только один, так как иначе это не будет сеть глубиной $1$. Следовательно $x$ может занять две различные позиции (либо $x$ останется на прежней позиции, либо поменяется местами с каким-нибудь элементом последовательности $a$).
\item $\textbf{Переход} \ i \rightarrow (i + 1)$: По индукционному предположению есть какая-то сеть глубины $i$, которая может поставить $x$ на различные $2^i$ мест. Тогда соединим компараторами каждое из этих $2^i$ различных мест, множество этих мест назовем $\mathbb{C}$, с местом, на которое $x$ еще не может быть поставлен, назовем множество этих мест $\mathbb{D}$. Этим действием мы получим сеть глубиной $i + 1$ (Так как соединяем нить, имеющую глубину $i$ с нитью глубиной $1$). Верно, что $\forall c1 \neq c2 \ c1, c2 \in \mathbb{C}$ $c1$ и $c2$ соединены с различными элементами множества $\mathbb{D}$($c1$ и $c2$ не могут быть соединены с одним и тем же элементом, так как такое сравнение не определено для компараторов). А так же, так как мы соединяем любой элемент множества $\mathbb{C}$ с элементом множества $\mathbb{D}$, то у каждого будет пара, а компаратор либо оставит $x$ на прежнем месте, либо переместит его. Следовательно количество мест, на которых может оказаться $x$ увеличится вдвое. А значит будет равно $2^i * 2 = 2^{(i + 1)}$.

Предположение доказано.
\end{enumerate}
По $\textbf{Утверждению 2}$ и доказанному утверждению о том, что сеть глубной $i$ может ставить элемент $x$ на $2^i$ различных мест в завивимости от входных данных, следует, что для объединения последовательности из $n - 1$-го элемента и $1$-го элемента сеть должна иметь глубину хотя бы $\log_2{n}$.

\end{document}