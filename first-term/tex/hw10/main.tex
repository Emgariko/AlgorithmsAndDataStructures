\documentclass{article}

\usepackage[utf8]{inputenc} 
\usepackage[russian]{babel} 
\usepackage{amsmath} 
\usepackage{hyperref} 
\usepackage{multicol}
\usepackage{graphicx}
\usepackage{amssymb}
\usepackage{tipa}

\title{Домашнее задание по АиСД №10}
\date{2019-12-1}
\author{Эмиль Гарипов M3138}

\begin{document}
\pagenumbering{gobble}

\maketitle
\newpage
\pagenumbering{arabic}
Введем обозначения для задач 1 и 4.\\\\
Пусть $V$ "--- множество всех вершин графа, $E$ - множество всех ребер графа. Значком $(x, y)$ будем обозначать ребро из $x$ в $y$. \\\\

\section*{Задача №1}
Заведем массив $ans$, где для каждого подмножества будем хранить длину кратчайшего цикла, который проходит по каждой вершине этого подмножества ровно оди раз. \\\\
Зафиксируем для каждого подмножества вершину, через которую будет проходить цикл. Это будет вершина из подмножества с наибольшим номером. 
\\\\
Будем перебирать эту вершину (назовем ее $f$) и соответствующее этой вершине подмножество исходного множества, назовем это подмножество $A$. Посчитаем следуюущую динамику: $dp[A][i]$ "--- минимальная длина простого пути из $f$ в $i$, проходящего по всем вершинам подмножества $A$. Для подсчета такой динамики проинициализируем $dp[\{f\}][f] = 0$. Далее переберем все подмножества, номера вершин которых меньше $f$. Назовем очередное такое подмножество $S$. Переберем еще и вершину, которая не вошла в $S$, но ее номер меньше $f$, назовем ее $nw$. Теперь чтобы посчитать значение $dp[S \cup \{nw\}][nw]$ необходимо перебрать вершину $x \in S: (x, nw) \in E$ и среди всех таких вершин выбрать ту, у которой значение $dp[S][x] + cost_{x}{nw}$ наименьшее, где $cost_{i}{j}$ "--- вес ребра (i, j).\\\\
На очередном шаге пересчета динамики (который описан выше), после перебора вершины $x$ и подсчета динамики, зная длины минимальных путей, который проходят из $f$ в $i$ и посещают все вершины подмножеств, которым соответствует вершина $f$ (номера всех вершин которых меньше  $f$) можно найти длины циклов, которые проходят через вершину $f$. Надо просто <<закольцевать>> путь, добавив к суммарному весу пути вес ребра из $f$ в вершину $x$ (мы соединяем ребром вершины $f$ и $x$, если таковое ребро $(f,x)$ имеется): $$ans[A] = min(ans[A], dp[x][A] + cost_{f}{x})$$\\\\
Таким образом, мы сначала перебираем <<стартовую вершину>>, в которой будет начинаться наш путь и продолжаться в вершинах, номера которых меньше стартовой. Далее перебираем все подмножества, в которых вершина с наибольшим номером "--- стартовая вершина. Затем считаем динамику и попутно обновляем $ans$ для текущего подножества. Итого для одной <<стартовой вершины>> это работает за $\mathcal{O}(2^{k}k^2)$, где $k$ "--- номер стартовой вершины. Получаем ассимптотику $\sum_{k = 0}^{n}k^{2}2^{k} = \mathcal{O}(n^{2}2^{n})$.
\\
\subsection*{Время работы}
Докажем это равенство. 
Нужно доказать, что $\sum_{k = 0}^{n}k^{2}2^{k} = \mathcal{O}(n^{2}2^{n})$. По опредению O-большого это значит, что $\exists c > 0, N \in \mathbb{N} : \forall \ n > N : \sum_{k = 0}^{n}k^{2}2^{k} \leq c \cdot n^{2}2^{n}$. \\\\
Оценим эту сумму сверху:$\sum_{k = 0}^{n}k^{2}2^{k} \leq (\sum_{k = 0}^{n}2^{k})\cdot n^{2} = 2^{k + 1}\cdot n^{2}$\\ Для выполения определения O-большого необходимо: $2^{k + 1}\cdot n^{2} \leq c \cdot n^{2}2^{n}$
Получаем $c \geq 2$. Ч.т.д.
\section*{Задача №4}
Посчитаем следующую динамику: $dp[A][i]$ "--- кол-во гамильтоновых путей, проходящих по всем вершинам подмножества всех вершин $A$, заканчивающихся в вершине под номером $i$, начинающихся в вершине $x$, такой, что $x$ "--- вершина с наибольшим номером среди всех вершин из подмножества $A$.\\\\
Тогда $\forall i$ верно (это следует из определения состояния динамики) : $dp[\{i\}][i] = 1$.\\\\
Далее необходимо пересчитать значения для всех подмножеств. Зная значения для подмножеств размера $cnt$, можно посчитать значения для подмножеств размера $cnt + 1$. Для этого просто нужно добавить в путь еще одну вершину, которой не было в пути. \\Справедлива следующая формула: 
\begin{equation}
dp[A][i] = \sum_{j \in V, (j, i) \in E}dp[A\setminus \{i\}][j]
\end{equation} где $A$ "--- подмножество размера хотя бы 2, $i \in A$, $i$ имеет не наибольший номер в подмножестве $A$.\\ Для подсчета всех путей, начинающихся в вершине с наибольшим номером среди всех вершин из $A$, заканчивающихся в вершине $i$ мы просто просуммировали количество всех путей, к которым можно добавить вершину $i$ (разумеется при наличии ребра в $i$ из последней вершины пути) таким образом, чтобы вершина $i$ имела не наибольший номер среди всех вершин на полученном пути. Последнее требование важно для выполнения определения состояния динамики. Ведь определение состояния динамики требует, чтобы путь начинался в вершине с наибольшим номером.\\\\
Теперь для получения ответа необходимо лишь просуммировать количество путей, конец и начало которых соединены ребром. Циклы, отличающиеся циклическим сдвигом мы посчитаем один раз, так как в определении состояния динамики мы требуем, чтобы путь начинался в вершине с наибольшим номером пути, говоря в определении динамики что путь идет из вершины с наибольшим номером в вершину $i$. Мы дважды посчитаем циклы, отличающиеся разворотом, поэтому необходимо разделить на 2.\\\\
$$ans = \frac{\sum_{i \in V, |A| \geq 2, (j, i) \in E}dp[A][i]}{2}$$ где $j$ "--- вершина с наибольшим номером из $A$. 

Видно (для подсчета формулы (1) необходимо перебрать все подмножества за $\mathcal{O}(2^{n})$, для каждого подмножества перебрать вершину $i$ за $\mathcal{O}(n)$, и для каждой такой вершины посчитать сумму за $\mathcal{O}(n)$) , что подсчет динамики работает за $\mathcal{O}(2^{n} \cdot n^{2})$ и использует $\mathcal{O}({2^{n} \cdot n})$ памяти.
\end{document}